\documentclass[a4paper,11pt,final]{article}
\usepackage[american]{babel}
\usepackage[T1]{fontenc}
\usepackage{graphicx}
\usepackage[svgnames]{xcolor}
\usepackage{amsmath}
\usepackage[utf8]{inputenc}
\usepackage[pdfborder={0 0 0}]{hyperref}
\usepackage{mathptmx}
\hypersetup{
    pdftitle={CamSim Physically-Based Rendering Model},
    pdfauthor={Martin Lambers}
} 

\begin{document}

\section*{CamSim illumination model}

This describes the illumination model (direct and single-bounce indirect) used
in CamSim.

It is based on this paper:
  M. Lambers, S. Hoberg, A. Kolb:
  Simulation of Time-of-Flight Sensors for Evaluation of Chip Layout Variants.
  In IEEE Sensors Journal, 15(7), 2015, pages 4019-4026.

The extension for single-bounce indirect illumination is loosely based on this paper:
  D. Bulczak, M. Lambers, A. Kolb:
  Quantified, Interactive Simulation of AMCW ToF Camera Including Multipath
  Effects.
  In MDPI Sensors, 18(1), 2018, pages 1424-8220.

\section{Symbols}

\subsection{Radiometric}

$Q$: Radiant energy $[J]$\\
$P$: Radiant flux (``power'') $[W]$\\
$I$: Radiant intensity $[W/\mathrm{sr}]$\\
$L$: Radiance (``radiant flux per unit solid angle per unit projected area'') $[W/\mathrm{sr}/m^2]$\\
$E$: Irradiance $[W/m^2]$

\subsection{Geometric}

$L$: Light source position\\
$C$: Camera center position\\
$P$: Point on object surface seen by camera pixel\\
$\widehat{n_P}$: Surface normal at $P$\\
$\theta_{P\rightarrow L}$: Angle between $\widehat{n_P}$ and $\widehat{L-P}$\\
$Q$: Another point (on another surface) that may act as a virtual point light\\
$\widehat{n_Q}$: Surface normal at $Q$\\
$\theta_{Q\rightarrow L}$: Angle between $\widehat{n_Q}$ and $\widehat{L-Q}$

\subsection{Other}

$f_P(\widehat{L-P}, \widehat{n_P}, \widehat{C-P})$: BRDF for $L\rightarrow P\rightarrow C$\\
$f_Q(\widehat{L-Q}, \widehat{n_Q}, \widehat{P-Q})$: BRDF for $L\rightarrow Q\rightarrow P$\\


\section{Direct Illumination}

$I_{L\rightarrow P} = \frac{P_L}{4\pi}$ (if $L$ is point light; there are alternatives)\\
$L_{L\rightarrow P} = \frac{I_{L\rightarrow P}}{d^2_{L\rightarrow P}}$\\
$L_{P\rightarrow C} = f_P(\widehat{L-P}, \widehat{n_P}, \widehat{C-P}) L_{L\rightarrow P} \cos(\theta_{P\rightarrow L})$\\
$E_C = L_{P\rightarrow C}$ ~~~(Note: no factor $\cos(\theta_{C\rightarrow P})$ here)\\
$P_C = E_C \cdot \mathrm{SensorPixelArea}$\\
$Q_C = P_C \cdot \mathrm{SignalDutyCycle} \cdot \mathrm{ExposureTime}$


\section{Indirect Illumination via Virtual Point Lights}

Start with the Rendering Equation:\\
$\displaystyle L_{P\rightarrow C} = \int_\Omega f_P(\omega_i, \widehat{n_P}, \widehat{C-P}) L_{i\rightarrow P} \cos(\theta_i) d\omega_i$

\noindent
Approximate this by splitting into direct and indirect parts, and approximating
the indirect part with single-bounce RSM VPLs:\\
$\displaystyle L_{P\rightarrow C} = f_P(\widehat{L-P}, \widehat{n_P}, \widehat{C-P}) L_{L\rightarrow P} \cos(\theta_{P\rightarrow L})
 + \frac{1}{|\mathrm{RSM}|}\sum_{Q\in \mathrm{RSM}} f_P(\widehat{Q-P}, \widehat{n_P}, \widehat{C-P}) L_{Q\rightarrow P} \cos(\theta_{P\rightarrow Q})$

\noindent
The direct part is already computed in the direct step as $L_{P\rightarrow C}$. Add the indirect part to it:\\
$\displaystyle L^*_{P\rightarrow C} = L_{P\rightarrow C}
 + \frac{1}{|\mathrm{RSM}|}\sum_{Q\in \mathrm{RSM}} f_P(\widehat{Q-P}, \widehat{n_P}, \widehat{C-P}) L_{Q\rightarrow P} \cos(\theta_{P\rightarrow Q})$

\noindent
using
$L_{Q\rightarrow P} = f_Q(\widehat{L-Q}, \widehat{n_Q}, \widehat{P-Q}) L_{L\rightarrow Q} \cos(\theta_{Q\rightarrow L})$\\
and
$L_{L\rightarrow Q} = \frac{I_{L\rightarrow Q}}{d^2_{L\rightarrow Q}}, \quad I_{L\rightarrow Q} = \frac{P_L}{4\pi}$


\section{Values to Store in RSM}

\begin{itemize}
\item Parameters that allow BRDF sampling at $Q$, e.g. $k_d, k_s, s$ for modified Phong
\item Position of $Q$ (to perform shadow test and to compute $\widehat{Q-P}$,
$\theta_{P\rightarrow Q}$, $\widehat{L-Q}$, $\widehat{P-Q}$,$ \theta_{Q\rightarrow L}$)
\item $\widehat{n_Q}$
\item $L_{L\rightarrow Q}$
\end{itemize}

%\section{Notes}
%
%\begin{itemize}
%\item Visibility tests:
%    \begin{itemize}
%    \item $P\rightarrow C$: rasterization does this for us
%    \item $L\rightarrow P$: per shadow map test
%    \item $L\rightarrow Q$: rasterization of the RSM ensures that only lighted points are in the RSM; this visibility test is implicit
%    \item $Q\rightarrow P$: not included in original RSM approach, BUT:
%        \begin{itemize}
%        \item Project $P$ into RSM of $Q$
%        \item Cast ray from $Q$ to $P$ and traverse it \textbf{in Q's RSM space}
%        \item Find obstacles based on depth values
%        \end{itemize}
%    \end{itemize}
%\end{itemize}

\end{document}
